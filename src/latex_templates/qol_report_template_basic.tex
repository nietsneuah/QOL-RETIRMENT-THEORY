\documentclass[11pt,a4paper]{article}
\usepackage[utf8]{inputenc}
\usepackage[T1]{fontenc}
\usepackage{geometry}
\geometry{margin=1in}

% Basic packages for BasicTeX compatibility
\usepackage{amsmath, amsfonts, amssymb}  % Mathematics
\usepackage{graphicx}                    % Graphics inclusion
\usepackage{array}                       % Extended array and tabular
\usepackage{xcolor}                      % Colors

% Define custom colors (compatible with BasicTeX)
\definecolor{qolblue}{RGB}{52, 73, 94}
\definecolor{qolgreen}{RGB}{39, 174, 96}
\definecolor{qolred}{RGB}{231, 76, 60}
\definecolor{qolgray}{RGB}{127, 140, 141}

% Title formatting
\title{\textbf{\textcolor{qolblue}{TITLE_PLACEHOLDER}}}
\author{QOL Retirement Framework Analysis}
\date{\today}

\begin{document}

% Title page
\maketitle
\thispagestyle{empty}

\vspace{2cm}

% Executive Summary Box
\begin{center}
\fcolorbox{qolblue}{lightgray!20}{
\begin{minipage}{0.9\textwidth}
\centering
\textbf{\Large Executive Summary}

\vspace{0.5cm}

\begin{flushleft}
\textbf{Analysis Overview:} SUMMARY_PLACEHOLDER

\vspace{0.3cm}

\textbf{Key Findings:}
\begin{itemize}
\item Portfolio success rate: \textbf{SUCCESS_RATE_PLACEHOLDER}
\item Average utility improvement: \textbf{IMPROVEMENT_PLACEHOLDER}
\item Risk assessment: \textbf{RISK_LEVEL_PLACEHOLDER}
\end{itemize}

\textbf{Recommendation:} RECOMMENDATION_PLACEHOLDER
\end{flushleft}
\end{minipage}
}
\end{center}

\newpage

% Table of Contents
\tableofcontents
\newpage

% Main Content

\section{Portfolio Analysis Results}

This report presents the results of comprehensive Quality of Life (QOL) framework analysis for retirement portfolio optimization. The analysis incorporates dynamic withdrawal strategies, market volatility modeling, and longevity risk assessment.

\subsection{Analysis Parameters}

The following parameters were used for this analysis:

\begin{itemize}
\item \textbf{Starting Portfolio Value:} \$STARTING_VALUE_PLACEHOLDER
\item \textbf{Starting Age:} STARTING_AGE_PLACEHOLDER years
\item \textbf{Analysis Horizon:} HORIZON_YEARS_PLACEHOLDER years  
\item \textbf{Monte Carlo Simulations:} N_SIMULATIONS_PLACEHOLDER
\item \textbf{Withdrawal Strategy:} WITHDRAWAL_STRATEGY_PLACEHOLDER
\item \textbf{Return Volatility:} RETURN_VOLATILITY_PLACEHOLDER
\end{itemize}

\subsection{Key Results Summary}

\begin{center}
\begin{tabular}{|l|c|}
\hline
\textbf{Metric} & \textbf{Value} \\
\hline
Portfolio Success Rate & SUCCESS_RATE_DETAILED_PLACEHOLDER \\
\hline
Median Final Portfolio Value & \$MEDIAN_FINAL_VALUE_PLACEHOLDER \\
\hline
Total Utility Generated & TOTAL_UTILITY_PLACEHOLDER \\
\hline
Average Annual Withdrawal & \$AVERAGE_WITHDRAWAL_PLACEHOLDER \\
\hline
Depletion Risk & DEPLETION_RISK_PLACEHOLDER \\
\hline
\end{tabular}
\end{center}

\section{Enhanced QOL Framework Benefits}

The Hauenstein QOL Framework provides several key advantages over traditional withdrawal strategies:

\subsection{Dynamic Quality of Life Adjustments}

The framework incorporates age-based quality of life adjustments that recognize the diminishing marginal utility of money as retirees age. This allows for higher consumption in early retirement years when physical health and activity levels are typically higher.

Key benefits include:
\begin{itemize}
    \item Higher early retirement spending when most beneficial
    \item Improved total lifetime utility compared to fixed withdrawal strategies  
    \item Portfolio sustainability through reduced late-life withdrawals
    \item Average utility improvement across scenarios: \textbf{TOTAL_IMPROVEMENT_PLACEHOLDER\%}
    \item Enhanced framework provides comprehensive risk assessment with depletion timeline analysis
    \item Sensitivity analysis confirms robustness across different market conditions
\end{itemize}

% Individual scenario details (to be generated by Python loop)
SCENARIO_DETAILS_PLACEHOLDER

\section{Portfolio Performance Visualizations}

The following charts provide comprehensive visual analysis of portfolio performance across all simulation scenarios:

\subsection{Key Performance Charts}

PERFORMANCE_CHARTS_PLACEHOLDER

% Depletion Analysis Section
\section{Portfolio Depletion Risk Analysis}

The enhanced framework includes comprehensive depletion timeline analysis, providing deeper insights into portfolio sustainability risk.

\subsection{Depletion Methodology}
DEPLETION_METHODOLOGY_PLACEHOLDER

\subsection{Risk Assessment Results}
DEPLETION_RESULTS_PLACEHOLDER

% Sensitivity Analysis Section  
\section{Parameter Sensitivity Analysis}

Understanding how key parameters affect outcomes is crucial for robust retirement planning. The enhanced framework includes comprehensive sensitivity analysis capabilities.

SENSITIVITY_ANALYSIS_PLACEHOLDER

% Methodology Section
\section{Technical Methodology}

\subsection{Mathematical Framework}

\subsubsection{Quality of Life Decay Function}

The QOL framework uses an age-based decay function to model the diminishing marginal utility of money:

\begin{equation}
QOL(age) = \max(0.4, 1.0 - 0.02 \times (age - 65))
\end{equation}

This function ensures that withdrawal rates decrease with age, reflecting reduced spending capacity and utility in later years.

\subsubsection{Dynamic Withdrawal Rate}

Annual withdrawal rates are calculated as:

\begin{equation}
W_t = P_t \times r_{base} \times QOL(age_t)
\end{equation}

Where:
\begin{itemize}
\item $W_t$ = Withdrawal amount in year t
\item $P_t$ = Portfolio value in year t  
\item $r_{base}$ = Base withdrawal rate (4\%)
\item $QOL(age_t)$ = Quality of life factor for age in year t
\end{itemize}

\subsubsection{Asset Allocation Strategy}

The framework uses age-based dynamic asset allocation:

\begin{equation}
\text{Equity Allocation} = \max(0.2, 1.1 - 0.01 \times age)
\end{equation}

\subsection{Monte Carlo Simulation}

The analysis uses Monte Carlo methods to model:
\begin{itemize}
\item Market return variability (normal distribution with mean 7\%, volatility 15\%)
\item Inflation variability (normal distribution with mean 2.5\%, volatility 1\%)
\item Sequence of returns risk
\item Longevity uncertainty
\end{itemize}

\subsection{Risk Metrics}

Key risk metrics calculated include:
\begin{itemize}
\item Portfolio depletion probability
\item Survival rates at ages 80, 90, and 100
\item Value at Risk (5\% worst-case scenarios)
\item Expected shortfall (average of worst 5\% outcomes)
\end{itemize}

\section{Conclusions and Recommendations}

\subsection{Key Findings}

CONCLUSIONS_PLACEHOLDER

\subsection{Implementation Recommendations}

RECOMMENDATIONS_PLACEHOLDER

\subsection{Risk Considerations}

\begin{itemize}
\item Market volatility can significantly impact outcomes
\item Sequence of returns risk is elevated in early retirement years
\item Longevity risk requires conservative planning assumptions
\item Regular portfolio rebalancing is essential for maintaining target allocation
\end{itemize}

\section{Appendix}

\subsection{Model Assumptions}

\begin{itemize}
\item \textbf{Market Returns:} 7\% average annual return (inflation-adjusted)
\item \textbf{Market Volatility:} 15\% annual volatility
\item \textbf{Inflation:} 2.5\% average with 1\% volatility
\item \textbf{Rebalancing:} Annual rebalancing to target allocation
\item \textbf{Fees:} Investment fees are not explicitly modeled
\end{itemize}

\subsection{Limitations}

\begin{itemize}
\item Model assumes consistent implementation of the QOL strategy
\item Tax implications are not considered in this analysis
\item Healthcare cost inflation may exceed general inflation assumptions
\item Individual utility functions may vary from the assumed QOL decay pattern
\end{itemize}

\end{document}